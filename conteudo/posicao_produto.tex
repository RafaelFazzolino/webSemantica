
\begin{table}[H]
\centering
	\begin{tabular}{|c|c|}
		\hline
		\cellcolor[HTML]{FFFFFF}{\color[HTML]{000000} \textbf{Para}} & Todo o corpo discente da Universidade de Brasília.                                                                                                   \\ \hline
		\textbf{Que}                                                 & Deseja selecionar as disciplinas mais adequadas para o seu contexto.                                                                                 \\ \hline
		\textbf{Os}                                                  & Alunos                                                                                                                                               \\ \hline
		\textbf{Que}                                                 & \begin{tabular}[c]{@{}c@{}}Selecionam disciplinas indesejadas ou até inúteis para o seu contexto \\ de atuação.\end{tabular}                         \\ \hline
		\textbf{Ao Contrário de}                                     & Selecionar apenas disciplinas que agreguem valor a sua formação.                                                                                     \\ \hline
		\textbf{Nosso Produto}                                       & \begin{tabular}[c]{@{}c@{}}Apresentará as melhores opções de matrícula para o Aluno, \\ de acordo com seus interesses e contexto atual.\end{tabular} \\ \hline
	\end{tabular}
	\caption{Sentença de Posição do Produto}
\label{tab:sentenca}
\end{table}

\subsection{Usuários} % (fold)
\label{sub:usu_rios}

	
Como o objetivo do projeto está no contexto de Universidades, mais especificamente, no contexto da Universidade de Brasília - UnB Gama, os usuários que terão contato com a solução são pessoas que frequentam a Universidade. Os Usuários e sua descrição podem ser observados na tabela a seguir:

\begin{table}[H]
	\centering
	\label{descUser}
		\begin{tabular}{lllll}
		\cline{1-2}
		\multicolumn{1}{|l|}{\textbf{Usuário}} & \multicolumn{1}{l|}{\textbf{Descrição}}                                                                                                                                                                                                                                                                                                                                                     & \textbf{} &  &  \\ \cline{1-2}
\multicolumn{1}{|l|}{Professor}        & \multicolumn{1}{l|}{\begin{tabular}[c]{@{}l@{}}Usuário que possui como interesse conhecer as relações entre ementas de \\ disciplinas para poder garantir maior conhecimento sobre o contexto do conteúdo\\ dado em sala. Conhecer essas relações garante que o professor poderá colaborar com\\ alunos indecisos ou até utilizar isso para conhecimento próprio do conteúdo.\end{tabular}} &           &  &  \\ \cline{1-2}
\multicolumn{1}{|l|}{Aluno}            & \multicolumn{1}{l|}{\begin{tabular}[c]{@{}l@{}}Usuário que possui como interesse conhecer as Ementas, os Pré-Requisitos e as relações\\ entre ementas das disciplinas. A utilização do sistema apoiará a escolha de disciplinas do\\ Aluno, de acordo com seus interesses e importância.\end{tabular}}                                                                                      &           &  &  \\ \cline{1-2}
\multicolumn{1}{|l|}{Outros}           & \multicolumn{1}{l|}{\begin{tabular}[c]{@{}l@{}}Usuário que possui como interesse conhecer os Cursos, Ementas e e Relações entre\\ disciplinas de um campus formado unicamente por Engenharias.\end{tabular}}                                                                                                                                                                                                                                                                                         &           &  &  \\ \cline{1-2}

		\end{tabular}
		\caption{Descrição dos Usuários}
\end{table}
% subsection usu_rios (end)

\subsection{Premissas} % (fold)
\label{sub:premissas}

\begin{itemize}
	\item \textit{Informações referentes às ofertas e fluxos estarão disponíveis para consulta}:

		A fonte de pesquisa utilizada será o Matrícula Web a partir da utilização de \textit{Web Scraping}, desse modo, a fonte de dados estará disponível sempre que o sistema do Matrícula Web estiver online.

	\item \textit{Será possível adicionar funcionalidades ao Portal FGA}:

		Como o Portal FGA é mantido pelos próprios alunos da Universidade de Brasília - UnB Gama, não existe muita burocracia para alteração ou inclusão de novas funcionalidades.

	\item \textit{Será possível alterar a estrutura das informações da Universidade representadas no Portal FGA}.
\end{itemize}

% subsection premissas (end)

\subsection{Restrições} % (fold)
\label{sub:restri_es}

\begin{itemize}
	\item Acesso limitado as informações;
	\item Escopo muito grande, sendo necessário primeiramente restringir à FGA;
	\item As informações a serem utilizadas pelo sistema são atualizadas com frequência, e afetam muito a funcionalidade.
\end{itemize}

% subsection restri_es (end)