
\subsection{Objetivos Gerais}

Transformação da aprendizagem em um processo autônomo e dinâmico, permitindo que os alunos construam seu histórico de acordo com seus objetivos, direcionando sua formação especifica. Para isso é necessário que os estudantes tenham o máximo de informações possíveis acerca dos conhecimentos prévios esperado em cada uma das disciplinas, visto que na falta de tais informações, alunos acabar por solicitar disciplinas aos quais não está preparado ou que são dos seus interesses, gerando como consequência o trancamento, ou até mesmo o abandono delas. A partir da elaboração de uma ontologia, conteúdos educacionais das disciplinas da matriz curricular dos cursos de engenharia podem ser representados e organizadas, fornecendo a descrição dos conceitos e as relações existentes em um dado domínio para o compartilhamento e entendimento comum. A ontologia vem para ajudar a elucidar as escolhas dos alunos.

\subsection{Objetivos Específicos}

	\begin{itemize}
		\item Tornar o processo de aprendizagem autônomo e dinâmico.

		\item Elaborar uma ontologia das disciplinas presentes no campus.
	\end{itemize}