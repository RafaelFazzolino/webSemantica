
\subsection{Objetivos Gerais}

Transformação da aprendizagem em um processo autônomo e dinâmico, permitindo que os alunos construam seu histórico de acordo com seus objetivos, direcionando sua formação especifica. Para isso é necessário que os estudantes tenham o máximo de informações possíveis acerca dos conhecimentos prévios esperado em cada uma das disciplinas, visto que na falta de tais informações, alunos acabar por solicitar disciplinas aos quais não está preparado ou que são dos seus interesses, gerando como consequência o trancamento, ou até mesmo o abandono delas. A partir da elaboração de uma ontologia, conteúdos educacionais das disciplinas da matriz curricular dos cursos de engenharia podem ser representados e organizadas, fornecendo a descrição dos conceitos e as relações existentes em um dado domínio para o compartilhamento e entendimento comum. A ontologia vem para ajudar a elucidar as escolhas dos alunos.

O projeto de construção do sistema de representação semântica no Portal FGA tem como objetivo verificar a viabilidade da utilização do Matrícula Web como fonte de dados sobre as disciplinas e cursos da UnB - Gama. Possui como objetivo, ainda, desenvolver uma arquitetura que apoie o sistema de representação semântica, utilizando Ruby on Rails.

A busca de ontologias existentes e referentes ao contexto estudado também é um dos grandes objetivos do projeto, já que primeiramente deve-se procurar ferramentas que auxiliem o projeto, caso não exista nenhuma, a única forma é desenvolver a ferramenta.

\subsection{Objetivos Específicos}

	\begin{itemize}
		\item Desenvolver a especificação de requisitos do sistema que apoiará a ontologia.

		\item Elaborar uma ontologia das disciplinas presentes no campus.

		\item Viabilizar a integração dos sistemas Portal FGA e Matrícula Web

		\item Buscar ontologias que possam embasar a construção da Ontologia proposta.

	\end{itemize}