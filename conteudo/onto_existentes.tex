
Existem iniciativas para a utilização de web semântica na representação de informações a respeito da estrutura organizacional de universidades, como por exemplo o \href{http://linkeduniversities.org/lu/index.php/vocabularies/index.html}{\textbf{Linked Universities}} . Esta organização lista as universidades no mundo que representam suas informações em formato linked data, e também todas os vocabulários hoje utilizadas para este tipo de representação, incluindo ontologias e fontes de terminologia.

Dentre as ontologias utilizadas, pode-se destacar a desenvolvidas por \href{http://www.epimorphics.com/public/vocabulary/org.html}{\textbf{Dave Reynolds}}, que foca em estruturas organizacionais genéricas. Também, existe a estrutura proposta por \href{http://vocab.org/aiiso/schema}{\textbf{Rob Styles e Nadeem Shabir}}, que ainda traz o ponto de vista organizacional, no entanto do ponto de vista de instituições acadêmicas, se aproximando mais do contexto proposto neste presente trabalho.

Adicionalmente, também vale indicar o trabalho desenvolvido por \href{https://www.cs.umd.edu/projects/plus/SHOE/onts/univ1.0.html}{\textbf{Jeff Heflin}}, que consiste em um rascunho de uma ontologia semelhante às anteriores e que ainda inclui informações a respeito de publicações. 