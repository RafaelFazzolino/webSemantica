\chapter*{Apêndice}

\begin{enumerate}
	\item \textbf{Entrevista Prof. Dr. Sergio Freitas}

		A questão do matricula web propriamente dito que cuida é o CPD, então o CPD mantém toda o banco de dados, toda linguagem, todo processo mesmo, todo desenvolvimento é CPD, não tem nada com nenhuma outra pessoa ou órgão na instituição. Basicamente o matricula web é uma ponta da parte acadêmica, e dentro da parte acadêmica há toda uma modelagem de banco de dados para modelar currículos, para modelar diversos projetos pedagógicos e etc. O que nós vemos no matricula web na verdade é a oferta referente a um dado período de tempo, com relação a um serie de disciplinas, casados, inicialmente, com um banco de professores, porque toda disciplina que está em determinada grade curricular ela vai ser alocada temporalmente em determinado momento para um professor e espacialmente para saber em qual sala está sendo dada a matéria.

		Depois, no segundo momento, quando há a matricula efetivamente, aí aloca-se dentro desse slot da turma, uma turma é uma disciplina da grade, com um professor com um espaço físico e um determinado horário. Dentro da turma há o matricula web, que nada mais é do que, dado já essa oferta, as pessoas vão entrar ali dentro, so que cada pessoa para entrar no matrícula web, que seriam os alunos, precisam também atender pré-requisitos, e os pré-requisitos estão relacionados as disciplinas que o aluno já fez, temporalmente em algum momento pra trás, dado o curso que ele está fazendo eu sei o que ele tem que fazer, então ele vai olhar e verificar quais são as disciplinas que ele já fez, se ele fez com aproveitamento, se não fez, quantas vezes ele fez, tudo isso está gravado no histórico daquele aluno, ai de tudo isso é feito um cálculo, pra poder saber se ele pode fazer ou não uma disciplina em que ele queira se pré-matrícula. Depois quando ele atende esse pré-requisito, chega um segundo pré-requisito que é classificação, aí são outras coisas, coisas distintas.

		Na minha opinião, o que vocês estão querendo fazer não está feito diretamente na matrícula web, está feito sim ao banco de dados que está por trás e que tem haver justamente com a modelagem do curso em si, qualquer curso que você está fazendo, porque dali que você quer na verdade, a única ponta que deixa margem a isso que vocês estão pensando, no meu entendimento, é a questão do TCC, porque o TCC acaba sendo uma instancia dinâmica e temporal que surge e que pode ser incorporada, não diretamente ao projeto pedagógico, mas na instancia de matricula. Apesar do no matricula web aparecer as informações, a formação daquilo não é ali e só vai servir para a fase final que é o aluno ir lá e falar, eu posso aqui eu posso ali. Então apesar da dificuldade da escolha que é real, o matricula web não trabalha direcionando para o perfil da pessoa, ele direciona para o perfil do curso. O que vocês estão pretendendo fazer é dizer o seguinte, eu gostaria de ser considerado, apesar de que existe algum tipo de consideração a nível do usuário, você não pode fazer isso porque você não tem pré-requisito, relações de dependência, pré-requisitos entre as disciplinas, mas isso é mais restrição, não é o contrário, mostrar o que seria bom você fazer, pelo menos ordena-las de tal forma que ela seja mais propícia para um dado perfil de aluno, ou seja, pra você, seria diferente pra ele, que seria diferente pra mim, mesmo que nós tivéssemos o mesmo leque de disciplinas disponíveis, identificação de perfil. Então vocês podem usar o material disponível do matricula web, no sentido de que, o que está lá é a oferta, tanto que seria só do semestre atual, você não consegue ver dos semestres anteriores, então aquilo não reflete exatamente o que é o curso de engenharia de software, porque só estão as ofertas desse período lá, começa por aí. Então, você pode pegar aquilo como rastro, para vocês usarem, para ficar mais fácil, mas tenham ciência de que aquilo não reflete o curso como um todo.

		O matricula web é muito limitado em relação ao que vocês estão sugerindo, mas tem informações lá que são uteis para vocês fazerem estudos, não há como ter acesso ao banco de dados, para ter acesso é muito complicado, mas temos acesso a um rastro temporal disso que é um semestre, então pode-se usar isso para montar a ontologia. O trabalho de vocês quer estabelecer uma relação entre duas disciplinas existentes de tal que vocês possam aconccelhar o aluno, dado um determinado, por isso que eu acho que dever ser por aluno e não só para o curso, para o curso, você tiver um trabalho de aconselhamento do curso, não dá para saber se é melhor ou pior, porque o projeto pedagógico foi construído baseado na cabeça de nos professora, que construímos o projeto pedagógico, baseado em outras experiências e ai se construiu um ciclo de disciplinas, que tem um fluxo obrigatório, optativas. Então tratando ao nível de disciplina, introduzindo a variável aluno, que é fácil para vocês, porque vocês são alunos, então o cliente são vocês mesmo, fica mais tranquilo, então deveria inserir aí o ganho do aluno e aí fazer as perguntas que vocês gostariam de fazer para mim, que eu imagino que é, como fazer para fazer isso? Quais são as informações que preciso ter p ter uma abordagem? Provavelmente as perguntas que vocês deveriam me enviar ou vão me enviar são pautadas nesse contexto por isso: Eu tenho uma disciplina aqui, vamos pensar em duas disciplinas, e aí eu estou pensando em botar a visão de aconselhamento para esse aluno.

		O problema seria eu ter um problema aqui embaixo e a pergunta que eu estou fazendo seria: eu fiz isso, eu tenho 3 disciplinas, quais delas é melhor eu fazer agora? Aí tem um problema de fácil compreensão e de importância crucial para vocês, agora o que eu tenho que fazer em termos de web semântica, e de ontologia de conceito. Bom primeiro ponto: no meu entendimento você tem que entender a modelagem da informação que o aluno vai fornecer para que você possa comparar com o que já existe. Porque um sistema de aconselhamento é o seguinte, não é só você falar assim, por que se você tirar o fator aluno eu vou ficar sempre aconselhando a mesma coisa. Isso não é um sistema de aconselhamento, é só uma página web estática já pronta que não aconselha nada. Então não muda nada de um semestre para o outro? Muda! Quais são as coisas que mudam, só os tópicos especiais? Não, muda ementa? Não a ementa não muda, mas muda professor! Por isso o aconselhamento seria sempre nesse sentido, não é bom porque não facilita para vocês... Apesar dos cursos da UnB possuírem no máximo 70\% de créditos obrigatórios acabam não aproveitando muito bem isso, porque os alunos se sentem desconfortáveis em como usar esse 30\%, o que causa esses problemas que vocês estão falando. Então, voltando, conhecendo as duas disciplinas, as ementas, as disciplinas que se ligam, qual seria a continuação natural? Isso já existe no Matricula Web para 1 semestre, e existe com força no projeto pedagógico que é a relação de dependência. Essa relação de dependência que parece ser obrigatória tem uma relação de conteúdo objetivo e se você aprendeu de forma objetiva o conteúdo da disciplina X1 você aprendera o conteúdo da disciplina X2, essa é a visão pedagógica.

		E no caso do aluno que quer pegar disciplinas optativa de uma área especifica? Conhecendo as disciplinas como estão hoje é muito difícil, de saber isso porque, ou elas estão elencadas formalmente \textit{alguém já definiu quais optativas seguir para concluir sua ênfase}. Isso não é um aconselhamento com base em ontologia, é só um banco de dados com um filtro. Porem eu acho que esse não é o foco, se eu for para o grande plano, se você introduzir o perfil dos usuários, você pode pensar nos conteúdos que ele já fez, os conteúdos que ele se saiu melhor, por que você pode pensar o seguinte, comigo você não gosta da disciplina, mas você pode estar fazendo com o do lado e tirar nota alta, é normal! Então o que indica isso, que você tem mais afinidade com um do que com outro. Esse conteúdo teoricamente se você tivesse que indicar uma das duas disciplinas, tendo que você executou as duas já, o aconselhamento mostra a possibilidade de fazer as duas, isso é ruim, já que o aconselhamento deveria ser p a disciplina que você se saiu melhor. Então isso é só uma regra que deveria ser seguida, já é um indicador.

		O outro indicador era você entrar dentro dos conteúdos das disciplinas, que vocês conhecem por ementa, e tentar olhar o texto e estabelecer uma relação com os textos das disciplinas seguintes. Como isso é feito? Na mão, ou você usa mecanismos automatizados para ver se as frases dos textos são relacionadas, aí você vai construindo uma ontologia de relacionamentos, mesmo que as duas disciplinas não estejam ligadas, porque não existe nada que diga que esta disciplina depende da outra, você está olhando para os conteúdos das duas e tentando achar um grau de relacionamento entre elas. Por exemplo eu acho que essa frase aqui se relaciona com essa frase, então a princípio você vai identificando quais seriam as disciplinas que tem um grau maior de relação, note que isso independe do aluno. Ai se você botar um peso nas duas frases ou 3 que identificou, você conseguiu categorizar essas frases que você identificou dentro daquilo que você quiser, exemplo segurança, se as frases estão categorizadas num banco de dados ou qualquer coisa aí você conseguisse categorizar automaticamente essas frases, aí você pode até ponderar se você olhar dá para disciplina dentro da UnB e der um pontinho a mais de acordo com o desempenho do aluno, aí você vai fazendo a ponderação.

		Aí depois feita esse cruzamento a respeito da vida do aluno, faz-se o aconselhamento por conta do perfil do aluno, e não por conta da disciplina. Esse é um algoritmo por do qual vocês conseguiriam identificar usando web semântica e ontologias, como você aconselharia um aluno a fazer ou escolher disciplinas. Mas eu só aconselharia a fazer o aconselhamento do semestre seguinte, o percurso completo de todas semestre é mais complicado, um algoritmo mais complexo!


\end{enumerate}