
A solução proposta permitiria a representação semântica das informações contidas no Matricula Web, integrando às informações contidas no Portal da FGA. Além de garantir a conformidade com linked data e prosseguir com as iniciativas de web semântica dentro da Universidade de Brasília, seria possível relacionar cursos, departamentos, e Trabalhos de Conclusão de Curso de acordo com a área de estudo. Também seria possível possibilitar a criação de um sistema de aconselhamento de disciplinas, de acordo com o perfil do estudante.

\subsection{Contexto} % (fold)
\label{sub:contexto}

O sistema proposto funcionaria como um intermediário entre o \href{https://fga.unb.br/}{\textbf{portal da FGA}} e o href{http://matriculaweb.unb.br}{\textbf{Matrícula Web}}. O Portal reúne informações a respeito da Faculdade do Gama, incluindo informações gerais a respeito dos cinco cursos (Engenharia de Software, Engenharia Automotiva, Engenharia de Energia, Engenharia Eletrônica e Engenharia Aeroespacial), como por exemplo corpo docente e descrição. Por outro lado, o Matrícula Web é a fonte primária de informações a respeito das ofertas, fluxos, e grades horárias. 

O Portal é desenvolvido com base na plataforma Noosfero, que por sua vez foi desenvolvido em linguagem Ruby. O Noosfero é um software livre, e mais informações a respeito do Portal podem ser encontradas na instância do GitLab no e no \href{https://softwarepublico.gov.br/}{\textbf{Portal do Software Público}}. No entanto, é mais difícil encontrar maiores informações a respeito do Matrícula Web, que é uma interface para um sistema de gestão de matrículas mais antigo, e cuja base de dados é de acesso restrito dentro da Universidade de Brasília.

Existe uma demanda elevada no aspecto de melhoria de representação das informações, sobretudo um campus de tecnologia que amplia continuamente o desenvolvimento de novas soluções. No entanto, existe a problemática do acesso às informações do Matrícula Web, já que a base de dados é restrita. Por outro lado, a integração dos conceitos de web semântica na plataforma utilizada para o desenvolvimento do portal é promissora, já que se trata de um software, com uma comunidade ativa, e em constante evolução.
% subsection contexto (end)