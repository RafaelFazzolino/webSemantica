
A solução proposta permitirá a representação semântica das informações contidas no Matricula Web, integrando às informações contidas no Portal da FGA. Além de garantir a conformidade com linked data e prosseguir com as iniciativas de web semântica dentro da Universidade de Brasília, seria possível relacionar cursos, departamentos, e Trabalhos de Conclusão de Curso de acordo com a área de estudo. Também será possível possibilitar a criação de um sistema de aconselhamento de disciplinas, de acordo com o perfil do estudante.

A obtenção dos dados ocorrerá a partir da utilização de Web Scraping no Matrícula Web. A técnica de Web Scraping é o termo utilizado para descrever a atividade de extrair dados de sites, formatando-os da forma desejada e guardando no Banco de Dados. Esta técnica será utilizada no Matrícula Web por motivos de privacidade do sistema, nos deixando sem acesso aos dados de outra forma.

Já no contexto do Portal FGA, os alunos da Universidade de Brasília possuem total acesso ao sistema, já que o mesmo é desenvolvido e mantido por alunos e professores do curso de Engenharia de Software. O Portal FGA é desenvolvido em linguagem Ruby e framework Rails, que também seria a linguagem utilizada para o desenvolvimento da API de Web Scraping.

\subsection{Ferramentas Utilizadas} % (fold)
\label{sub:ferramentas_utilizadas}
	
	Para o desenvolvimento de toda a solução, muitas ferramentas serão utilizadas. As mesmas estão descritas a seguir.

\subsubsection{Protégé}

	É a ferramenta líder de engenharia de ontologias, um editor livre e um sistema de aquisição de conhecimento. Ele fornece uma interface gráfica de usuário para definir ontologias, também inclui classificadores dedutivas para validar se os modelos são consistentes e inferir novas informações com base na análise de uma ontologia. Como Eclipse, Protégé é uma estrutura para que vários outros projetos sugerem plugins. 

	Este aplicativo é escrito em Java e utiliza pesadamente balanço para criar a interface do usuário. Protégé tem recentemente mais de 200.000 usuários registrados e está sendo desenvolvido na Universidade de Stanford , em colaboração com a Universidade de Manchester e está disponível sob a Licença Pública Mozilla 1.1.(Dragan Gašević; Dragan Djurić; Vladan Devedžić (2009). Model Driven Engineering and Ontology Development (2nd ed.). Springer. p. 194. ISBN 978-3-642-00282-3.)

\subsubsection{Ruby on Rails}

	Ruby on Rails é um \textit{framework} livre da linguagem de programação \textit{Ruby}, que promete aumentar velocidade e facilidade no desenvolvimento de sites orientados a banco de dados, uma vez que é possível criar aplicações com base em estruturas pré-definidas e com gerenciamento de memória automático. As aplicações criadas utilizando o framework Rails são desenvolvidas com base no padrão de arquitetura MVC(\textit{Model-View-Controller}).
% subsection ferramentas_utilizadas (end)

\subsubsection{Personalização das Recomendações} % (fold)
\label{ssub:personaliza_o_das_recomenda_es}

	De forma a oferecer recomendações mais precisas, espera-se poder treinar o sistema construindo perfis de preferência para cada usuário. Além de inferir relações de dependência a partir das disciplinas já cursadas, também serão usados questionários para definir preferências pessoais, que serão usadas como parâmetros adicionais para a recomendação.

	O questionário abaixo é um exemplo de uma série de questões que podem ser utilizadas para a criação de tal perfil.

	\begin{enumerate}
		\item Selecione a disciplina que demonstra a sua área interesse dentro de Engenharia de Software:

			\begin{enumerate}
				\item Engenharia de Requisitos
				\item Desenho de Software
				\item Fundamentos de Sistemas Operacionais
				\item Interação Humano-Computador
			\end{enumerate}

		\item Qual sua maior dificuldade dentro de Engenharia de Software?

			\underline{\hspace{13.8cm}}

			\underline{\hspace{13.8cm}}.

		\item Quantos créditos você deseja pegar no semestre?

			\begin{enumerate}
				\item 0 á 16;
				\item 16 á 22;
				\item 22 á 26;
				\item 26 a 32.
			\end{enumerate}

		\item Qual sua disponibilidade de horário?
			\begin{enumerate}
				\item Matutino;
				\item Vespertino;
				\item Ambos.
			\end{enumerate}

		\item Qual sua disponibilidade de tempo para estudo?
			\begin{enumerate}
				\item Pouco;
				\item Razoável;
				\item Muito.
			\end{enumerate}
		\item Você gosta de trabalhar em equipe?
			\begin{enumerate}
				\item Sim;
				\item Não.
			\end{enumerate}
		\item Você é proavito?
			\begin{enumerate}
				\item Sim;
				\item Não.
			\end{enumerate}
	\end{enumerate}

% subsubsection personaliza_o_das_recomenda_es (end)
