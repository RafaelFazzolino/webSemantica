
O desenvolvimento de conteúdo Web explodiu no mundo por volta da década de 90, onde todos os conteúdos eram desenvolvidos por programadores de \textit{software} com objetivo de informar leitores Humanos. Desta forma, mecanismos de busca como \textit{Google}, \textit{AltaVista} entre outros, ainda necessitam da intervenção humana para que se possa identificar as respostas que realmente atendem as nossas demandas. A internet atual é considerada como Web Sintática, ou \textit{web 2.0} e estamos caminhando para a web 3.0, pois nela os computadores fazem apenas a apresentação da informação a interpretação fica sob responsabilidade dos seres humanos.

O fato dos computadores não conseguirem interpretar essa informações se refere ao fato de que as páginas da web não têm informações sobre si mesmas, ou seja, que tipo de conteúdo está descrito e as informações sobre os assuntos na qual a página se refere.

Um bom exemplo do que estamos vivenciando hoje, neste contexto são os mecanismos de busca da \textit{Google}, entre outros, que contém um grande número de páginas encontradas, porém com uma precisão muito pequena. Para melhorar isso é necessário adicionar semântica a essas páginas.

A utilização de Semântica para o desenvolvimento de sistemas web torna possível a interpretação do conteúdo por parte do computador, facilitando a busca de conteúdo que agregue, realmente, valor para o usuário.


\subsection{Fundamentação Teórica}

No inicio, as páginas da web eram desenvolvidas por programadores de software, e todas essas páginas criadas eram informações direcionadas para leitores humanos. Desta forma, mecanismos de busca como Google, AltaVista entre outros, ainda necessitam da intervenção humana para que se possa identificar as respostas que realmente atendem as nossas demandas. A internet atual é considerada como web sintática ou web 2.0 e estamos caminhando para a web 3.0, pois nela os computadores fazem apenas a apresentação da informação a interpretação fica a cabo dos seres humanos.

O fato dos computadores não conseguirem interpretar essas informações esta no fato de que as páginas da web não tem informação sobre si mesma, ou seja, que tipo de conteúdo está descrito e as informações sobre os assuntos na qual a página se refere.

Um bom exemplo do que estamos vivenciando hoje, neste contexto são os mecanismos de busca da Google, entre outros, que contém um grande número de páginas encontradas, porém com uma precisão muito pequena. Para melhorar isso é necessário adicionar semântica a essas páginas.

A Web Semântica é uma extensão da \textit{World Wide Web} atual, que permitirá aos computadores e humanos trabalharem em cooperação, a web 1.0 possui como principal preocupação, a própria construção de rede, torná-la acessível e comercializável, criação de sites. Já a Web 2.0 é mais focada na colaboração \textit{online} e na partilha entre utilizadores, rede sociais, blogs e etc, a 3.0 é que consideramos hoje de web inteligente, é uma web que se baseia na capacidade de interpretar conteúdos em rede, trazendo resultados mais precisos e inteligentes, mais conhecida como web Semântica.

Em um artigo de Tim Berners Lee \cite{berners2001semantic}, James Hendler e Ora Lassila, eles apresentaram cenários futuros onde a web semântica possui um papel fundamental em facilitar tarefas do cotidiano das pessoas. A ideia central da web semântica é categorizar a informação de maneira padronizada, facilitando seu acesso, é uma ideia bem semelhante à classificação dos seres vivos, essa web deve ser o mais descentralizada possível, a previsão de Hendler é de que qualquer empresa, universidade ou organização na web do futuro terá seu modelo próprio de organização, tendo assim uma série de modelos de organização em paralelo.

Um dos termos que tem grande importância na web semântica são os \textit{metadados}, que são dados sobre dados, eles servem para indexar páginas e sites na web semântica, permitindo que outros computadores saibam de que assunto eles tratam, são informações sobre um certo dado como data, autor ou editora, que podem ser entendidos por máquinas.

Os metadados podem ser de 5 tipos, administrativo, descritivo, preservação, técnica e utilização, podemos também citar algumas características dos metadados, eles não precisam ser necessariamente digitais, podem ser obtidos a partir de uma variedade de fontes, evoluem durante a vida útil do sistema de informação ou objeto a que se refere e vão além de fornecer dados sobre um objeto.

Temos vários formatos de captura de metadados, entre eles, \textit{Duplin Core} que é composto por vários elementos como, assunto, titulo, criador, descrição do conteúdo do objeto, editor, entre outros, é um padrão bem simples, porém não apresenta um semântica tão expressiva, outro formato de captura de metadados é o \textit{Framework de Warwick} que surgiu pela necessidade de ampliar o \textit{Duplin Core}, essa estrutura é baseada no conceito de conteines, que agrega vários tipo de metadados em pacotes separados, o Dublin Core é apenas um desses pacotes, porém esse formato também tinha algumas falhas, então foi criado outro formato chamado RDF (\textit{Resource Description Framework}), o RDF é uma linguagem declarativa que fornece uma maneira padronizada de utilizar o XML para representar metadados no formato de sentenças sobre propriedades e relacionamentos entre itens na web. O RDF foi projetado de modo a representar metadados de recursos web de maneira legível e, sobretudo, processável por máquinas.

Jim Hendler acredita que no futuro cada site e aplicação na internet vão contar com sua própria ontologia de termos, construídas e mantidas por pessoas, entidades ou instituições independentes e no futuro, os serviços providos na internet como reservas em hotéis, compras e etc, poderão ser grandemente expandidas e melhoradas se for adicionado semântica aos presentes recursos, as pessoas passarão a utilizar agentes, programas de software autônomos que agem em beneficio de seus usuários, para realizar suas tarefas, porém quem ira tomar as decisões ainda somos nós, o seu papel será de reunir, organizar, selecionar e apresentar informações a um usuário humano, que tomará suas decisões, esses agentes farão uso das ontologias e metadados para realizar esses fins.

Vale ressaltar que web semântica não é uma Inteligencia Artificial, não é uma web separada e não vai exigir que todas as aplicações utilizem expressões complexas. A web semântica vai ter um impacto gigantesco no mundo já que vai beneficiar tanto o comercio eletrônico como os usuários que poderão ter mais tempo livres já que a essa web irá trabalhar por ele.